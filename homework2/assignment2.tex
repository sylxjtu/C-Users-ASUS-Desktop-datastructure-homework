\documentclass[UTF8]{ctexart}
\usepackage{amsmath}
\usepackage{graphicx}
\usepackage{minted}
\usepackage{hyperref}
\usepackage{fontspec}
\usepackage{float}

\hypersetup{hidelinks}
\graphicspath{ {images/} }
\setmonofont{Fantasque Sans Mono}
\setCJKmonofont{Microsoft YaHei}

\title{数据结构第二次作业}
\date{\today}
\author{沈俞霖}

\begin{document}
  \maketitle
  \vspace{100mm}
  \begin{flushright}
  \textbf{课程} \makebox[7em][l]{数据结构与算法}

  \textbf{姓名} \makebox[7em][l]{沈俞霖}

  \textbf{专业班级} \makebox[7em][l]{软件51}

  \textbf{学号} \makebox[7em][l]{2151601013}

  \textbf{邮箱} \makebox[7em][l]{sylxjtu@outlook.com}

  \textbf{提交日期} \makebox[7em][l]{\today}
  \end{flushright}
  \newpage

  \tableofcontents
  \newpage

  \section{实现List}
    \subsection{作业题目}
      未排序的顺序数组、已排序的顺序数组、未排序的单循环链表和已排序的单循环链表
      为了简化问题,这些数据结构都仅仅存储 int 类型,并且实现了如下抽象对象类型
      \inputminted{java}{src/sylxjtu/ListInterface.java}
    \subsection{程序实现}
    包括数据结构实现和测试代码,完整代码详见
      \href{https://github.com/sylxjtu/datastructure-homework/tree/master/homework2/src}{Github代码仓库}
      \footnote{https://github.com/sylxjtu/datastructure-homework/tree/master/homework2/src}
      \paragraph{无序数组}
      \inputminted{java}{src/sylxjtu/UnsortedArray.java}
      \paragraph{有序数组}
      \inputminted{java}{src/sylxjtu/SortedArray.java}
      \paragraph{无序链表}
      \inputminted{java}{src/sylxjtu/UnsortedLinkedList.java}
      \paragraph{有序链表}
      \inputminted{java}{src/sylxjtu/SortedLinkedList.java}
    \subsection{程序运行结果与时间统计}
      \paragraph{运行结果}
      所有程序均通过测试

    \subsection{算法分析}
      \begin{table}[h!]
      \centering
      \begin{tabular}{|l|l|l|l|l|}
      \hline
      数据结构        & \textbf{无序数组} & \textbf{有序数组} & \textbf{无序链表} & \textbf{有序链表} \\
      \hline
      search      & $O(n)$        & $O(\log n)$   & $O(n)$        & $O(n)$        \\
      insert      & $O(1)$        & $O(n)$        & $O(1)$        & $O(n)$        \\
      delete      & $O(n)$        & $O(n)$        & $O(n)$        & $O(n)$        \\
      successor   & $O(1)$        & $O(1)$        & $O(1)$        & $O(1)$        \\
      predecessor & $O(1)$        & $O(1)$        & $O(1)$        & $O(1)$        \\
      minimum     & $O(n)$        & $O(1)$        & $O(n)$        & $O(1)$        \\
      maximum     & $O(n)$        & $O(1)$        & $O(n)$        & $O(1)$        \\
      kthElement  & $O(n)$        & $O(1)$        & $O(n)$        & $O(k)$        \\
      \hline
      \end{tabular}
      \end{table}
  \section{创建布尔表达式计算器}
    \subsection{题目描述}
      本题是要计算如下的布尔表达式: $(T |T)\& F \&(F|T)$
      其中 $T$ 表示 True,$F$ 表示 False。
      表达式可以包含如下运算符:$!$表示 not,$\&$表示 and,$|$表示 or,允许使用括号。

      为了执行表达式的运算,要考虑运算符的优先级:not 的优先级最高,or 的优先级最低。
      计算器要产生 $V$ 或 $F$,表达最终表达式计算的结果。

      对输入的表达式的要求如下:
      \begin{enumerate}
        \item 一个表达式不超过 100 个符号,符号间可以用任意个空格分开,或者根本没有空格,所以表达式总的长度也就是字符的个数,它是未知的。
        \item 要能处理表达式中出现括号不匹配、运算符缺少运算操作数等常见的输入错误。
      \end{enumerate}
    \subsection{问题分析}
      表达式文法为

      \begin{equation}
      \begin{aligned}
        expr & \rightarrow andexpr | expr \\
             & \rightarrow andexpr
      \end{aligned}
      \end{equation}

      \begin{equation}
      \begin{aligned}
        andexpr & \rightarrow notexpr \& expr \\
                & \rightarrow notexpr
      \end{aligned}
      \end{equation}

      \begin{equation}
      \begin{aligned}
        notexpr & \rightarrow ! notexpr \\
                & \rightarrow atom
      \end{aligned}
      \end{equation}

      \begin{equation}
      \begin{aligned}
        atom & \rightarrow T \\
             & \rightarrow F \\
             & \rightarrow (expr) \\
      \end{aligned}
      \end{equation}

      规约为LL(1)文法为

      \begin{equation}
      \begin{aligned}
      expr &\rightarrow andexpr \  rexpr
      \end{aligned}
      \end{equation}

      \begin{equation}
      \begin{aligned}
      rexpr &\rightarrow | expr \\
            &\rightarrow \epsilon
      \end{aligned}
      \end{equation}

      \begin{equation}
      \begin{aligned}
      andexpr &\rightarrow notexpr \  randexpr
      \end{aligned}
      \end{equation}

      \begin{equation}
      \begin{aligned}
      randexpr &\rightarrow \& andexpr \\
               &\rightarrow \epsilon
      \end{aligned}
      \end{equation}

      \begin{equation}
      \begin{aligned}
      notexpr &\rightarrow ! notexpr \\
              &\rightarrow atom
      \end{aligned}
      \end{equation}

      \begin{equation}
      \begin{aligned}
      atom &\rightarrow T \\
           &\rightarrow F \\
           &\rightarrow (expr)
      \end{aligned}
      \end{equation}

      之后可以通过递归下降法求解

    \subsection{程序实现}
      \inputminted{java}{src/sylxjtu/BoolEquationEvaluate.java}

    \subsection{程序运行结果与时间统计}
      \paragraph{运行结果} 通过POJ2109测试
      \paragraph{时间统计} POJ2109测试,10组表达式,用时279ms

    \subsection{算法分析}
      根据递归下降法,每次扫描一个符号,故复杂度为$O(N)$($N$为符号数)

  \section{队列实现基数排序}
    \subsection{题目描述}
      利用队列实现对某一个数据序列的排序(采用基数排序),
      其中对数据序列的数据(第 1 和第 2 条进行说明)
      和队列的存储方式(第 3 条进行说明)有如下的要求:
      \begin{enumerate}
        \item 当数据序列是整数类型的数据的时候,数据序列中每个数据的位数不要求等宽,比如: \\
        1、21、12、322、44、123、2312、765、56
        \item 当数据序列是字符串类型的数据的时候,数据序列中每个字符串都是等宽的,比如: \\
        "abc","bde","fad","abd","bef","fdd","abe"
        \item 要求重新构建队列的存储表示方法:使其能够将 n 个队列顺序映射到一个数组listArray中,每个队列都表示成内存中的一个循环队列
      \end{enumerate}
    \subsection{问题分析}
      首先确定排序的顺序,对于整数排序的顺序是按数的大小升序,对于字符串排序的顺序是字典序升序
      按照要求3,通过一个循环数组实现队列的操作
    \subsection{程序实现}
      \paragraph{整数队列接口}
      \inputminted{java}{src/sylxjtu/QueueInt.java}
      \paragraph{整数队列实现}
      \inputminted{java}{src/sylxjtu/LoopArrayQueueInt.java}
      \paragraph{字符串队列接口}
      \inputminted{java}{src/sylxjtu/QueueString.java}
      \paragraph{字符串队列实现}
      \inputminted{java}{src/sylxjtu/LoopArrayQueueString.java}
      \paragraph{整数基数排序}
      \inputminted{java}{src/sylxjtu/IntegerRadixSort.java}
      \paragraph{字符串基数排序}
      \inputminted{java}{src/sylxjtu/StringRadixSort.java}
    \subsection{程序运行结果}
      \paragraph{运行结果} 通过两组样例测试
    \subsection{算法分析}
      \paragraph{复杂度分析} 基数排序复杂度为$O(kN)$ ,其中$k$为最大数据长度, k为数组大小
\end{document}
